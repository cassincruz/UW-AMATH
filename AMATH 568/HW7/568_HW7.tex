\documentclass[12pt, a4paper]{article}

\usepackage{fullpage}
\usepackage{latexsym}
\usepackage{amsfonts}
\usepackage{amssymb}
\usepackage{graphicx}
\usepackage{amsmath}
\usepackage{float}
\usepackage{subcaption}

\pagestyle{empty}

\begin{document}

\title{{AMATH 568\\
Advanced Differential Equations}\\
{\bf \Huge Homework 7}}

\author{Lucas Cassin Cruz Burke}

\date{Due: March 3, 2023}

\maketitle

\begin{enumerate}
    \item Consider the Optical Parametric Oscillator as given in Lecture 23 of the notes (pages 99-102).
    
    \begin{enumerate}
        \item Assuming slow time $\tau = \epsilon^2 t$ and slow space $\xi = \epsilon x$, derive the Fisher-Kolmogorov equation for the slow equation of the instability (the expression after Eq. (518)). 
        
        \textbf{Solution:} The Optical Parametric Oscillator equation is given by 
        
        \begin{align*}
            U_t &= \frac{i}{2} U_{xx} + VU^* - (1+i \Delta_1)U \\ V_t &= \frac{i}{2} \rho V_{xx} - U^2 - (\alpha + i\Delta_2)V + S
        \end{align*}

        The stable uniform steady-state response of the OPO is given by
        \begin{align*}
            U &= 0 \\ V &= \frac{S}{\alpha + i\Delta_2}
        \end{align*}

        It can be shown via linear stability analysis that this solution becomes unstable once the external pumping amplitude $|S|$ becomes greater than $|S_c|$, where $S_c$ is the critical pumping strength $$S_c = (\alpha + i\Delta_2)(1+i\Delta_1).$$ 

        We consider an OPO system with an external pumping term $S$ near $S_c$, which we asymptotically expand as 

        $$S = S_c + \epsilon^2 C + \mathcal O(\epsilon^3)$$

        where $C$ is a constant and $0 < \epsilon \ll 1$. Next, we expand about the steady-state solution by letting 

        \begin{align*}
            U &= \epsilon u(\tau, \xi) +\mathcal O(\epsilon^3)\\ 
            V &= \frac{S_c + \epsilon^2 C}{\alpha + i\Delta_2} + \epsilon^2 v(\tau, \xi) + \mathcal O(\epsilon^3) \\&= (1+i \Delta_1)(1+\epsilon^2 C/S_c) + \epsilon^2 v(\tau, \xi) + \mathcal O(\epsilon^3)
        \end{align*}

        where we have assumed the slow time $\tau = \epsilon^2 t$ and slow space $\xi = \epsilon x$. Plugging these in to the OPO equation and applying the chain rule $\partial_x \rightarrow \epsilon \partial_\xi$ and $\partial_t \rightarrow \epsilon^2 \partial_\tau$ results in the equations 

        \begin{align*}
            \epsilon^2 u_\tau &= \frac{i}{2} \epsilon^2 u_{\xi\xi} + u^* \left( \frac{S_c + \epsilon^2 C}{\alpha + i\Delta_2} + \epsilon^2 v \right) - (1+ i\Delta_1) u \\
            \epsilon^2 v_\tau &= \frac{i}{2} \rho \epsilon^2 v_{\xi\xi} - u^2 - (\alpha + i\Delta_2) v 
        \end{align*}

        which we can manipulate to the form 

        \begin{align}
            (1+i \Delta_1)(u^*-u) &= \epsilon^2 \left(\frac{i}{2} u_{\xi\xi} - u_\tau + vu^* + \frac{C}{\alpha + i \Delta_2} u^* \right) \\
            (\alpha + i\Delta_2)v &= -u^2 + \epsilon^2 \left( \frac{i}{2} \rho v_{\xi\xi} - v_\tau \right)
        \end{align}

        To leading order, (1) gives us 

        \begin{align*}
            (1+i \Delta_1) (u^* - u) &= \mathcal O(\epsilon^2) \\
            \Rightarrow \Im(u) \sim \epsilon^2
        \end{align*}
        
        Hence we conclude that $u$ is real to order $\mathcal O(1)$. Next, to leading order, (2) gives us 

        \begin{align*}
            (\alpha + i\Delta_2) v &= - u^2 \\
            \Rightarrow v &= \frac{-u^2}{\alpha + i \Delta_2}
        \end{align*}

        Using this we may calculate $vu^*$ to $\mathcal O(1)$ as 

        \begin{align*}
            vu^* &= \frac{-u^2u^*}{\alpha + i \Delta_2} = \frac{-|u|^2u}{\alpha + i \Delta_2} + \mathcal O(\epsilon^2)
        \end{align*}

        Using these expressions for $u^*$ and $vu^*$ we can write the right hand forcing of (1) as 

        \begin{align*}
            R &= \epsilon^2 \left(\frac{i}{2} u_{\xi\xi} - u_\tau - \frac{u^3}{\alpha + i \Delta_2} + \frac{C}{\alpha + i \Delta_2} u \right) + \mathcal O(\epsilon^4)
        \end{align*}


        Note that since the null space of the adjoint of the leading $\mathcal O(1)$ governing equation is the space of all real functions, by the Fredholm Alternative theorem $R$ must be purely imaginary. Hence, at $\mathcal O(\epsilon^2)$ we have the governing equation
        
        \begin{align}
            u_\tau = \frac{i}{2} u_{\xi\xi} + \frac{Cu - u^3}{\alpha+i\Delta_2}
        \end{align}

        We now perform the following coordinate transformations. Define the scaled function $\varphi$ by $u = \sqrt{\alpha + i \Delta_2} \varphi$, along with the scaled slow space $\zeta$ defined by $\zeta = \sqrt{\frac{2}{i}} \xi$. Additionally, define $\gamma = C/(\alpha + i\Delta_2)$. Then substituting $u \rightarrow \varphi$ and $\xi \rightarrow \zeta$ into the (2) gives us the \textbf{Fisher-Kolmogorov equation} for the slow equation of the instability.

        \begin{align*}
            \varphi_\tau = \varphi_{\zeta \zeta} + \gamma \varphi - \varphi^3
        \end{align*}

    \end{enumerate}
\end{enumerate}

\end{document}