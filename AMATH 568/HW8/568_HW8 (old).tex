\documentclass[12pt, a4paper]{article}

\usepackage{fullpage}
\usepackage{latexsym}
\usepackage{amsfonts}
\usepackage{amssymb}
\usepackage{graphicx}
\usepackage{amsmath}
\usepackage{float}
\usepackage{subcaption}

\pagestyle{empty}

\begin{document}

\title{{AMATH 568\\
Advanced Differential Equations}\\
{\bf \Huge Homework 8}}

\author{Lucas Cassin Cruz Burke}

\date{Due: March 8, 2023}

\maketitle

\begin{enumerate}
    \item Consider the inverted pendulum dynamics: $$y'' + (\delta + \epsilon \cos \omega t) \sin y = 0$$
    
    \begin{enumerate}
        \item Perform a Floquet analysis (computationally) of the pendulum with continuous forcing $\cos \omega t$. 
        
        \textbf{Solution:} We begin by Taylor expanding $\sin y$ around the unstable inverted equilibrium position $y \approx \pi$. 
        
        $$\sin(y+\pi) = -\sin y = -y + \frac{y^3}{3!} + \dots$$

        Then to first order we find the governing equation to be given by Matthieu's equation

        $$y'' - (\delta + \epsilon \cos \omega t)y=0$$

        Next, we shift the time variable by $t \rightarrow t + \pi/2$ to write this governing equation as 

        $$y'' - (\delta + \epsilon \sin \omega t)y=0$$

        Lastly, we approximate the sinusoidal forcing $\sin \omega t$ with a step-wise constant square-wave function with frequency $\omega$. With this change, the governing equations over one period $0 \le t <T$ with $T = \frac{2\pi}{\omega}$ are given by 

        \begin{align*}
            y'' - (\delta + \epsilon)y &= 0 && 0 \le t < T/2 \\
            y'' - (\delta - \epsilon) y &= 0 && T/2 \le t < T
        \end{align*}

        With this approximation we can calculate the Floquet discriminant for the inverted pendulum as 

        $$\Gamma = y_1(T) + y_2'(T)$$

        where $y_1(t)$ and $y_2(t)$ satisfy the initial conditions 

        \begin{align}
            y_1(0)= y_2'(0)=1 \\ y_2(0)= y_1'(0)=0
        \end{align}

        Using the square-wave approximation above, it is straightforward to compute the general solution $y(t)$ over the two half-periods to be 

        \begin{align*}
            y(t) = c_1 \exp \left[ \sqrt{\delta+\epsilon} t \right] + c_2 \exp \left[-\sqrt{\delta + \epsilon} t \right] && 0 \le t < T/2 \\
            y(t) = c_3 \exp \left[ \sqrt{\delta-\epsilon} t \right] + c_4 \exp \left[-\sqrt{\delta - \epsilon} t \right] && T/2 \le t < T
        \end{align*}

        Hence, to find the particular solutions $y_1(t)$ and $y_2(t)$ we must solve for the coefficients $c_i$ for which $y(t)$ satisfies the boundary conditions (1) and (2), and which is furthermore continuous with continuous first derivative at $t = T/2$. This can be done computationally in Mathematica by defining the functions $y_L(t)$ and $y_R(t)$ as the left and right-hand side general solutions 

        \begin{align*}
            y_L(t) &= c_1 \exp \left[\sqrt{\delta +\epsilon } t \right]+c_2 \exp \left[-\sqrt{\delta +\epsilon }t\right] \\
            y_R(t) &= c_3 \exp \left[\sqrt{\delta -\epsilon } \left(t-T/2\right)\right]+c_4 \exp \left[-\sqrt{\delta -\epsilon } \left(t-T/2\right)\right]
        \end{align*}

        where the right-hand function $y_R(t)$ is defined on the interval $t \in [0, T/2]$ such that for the full solution $y(t)$ we have $y(T)= y_R(T/2)$. We can then find $y_{1,L}(t)$ and $y_{2,L}(t)$ by solving for $c_1$ and $c_2$ such that $y_{1,L}$ and $y_{2,L}$ satisfy the initial conditions (1) and (2). Then we compute $y_{1,R}(t)$ and $y_{2,R}(t)$ by solving for $c_3$ and $c_4$ such that $y_{i,R}(0) = y_{i,L}(T/2)$ and ${y_{i,R}}'(0)={y_{i,L}}'(T/2)$. Executing this process in Mathematica, we find the solutions for the right-hand side of the particular solutions $y_1$ and $y_2$ to be

        \begin{align*}
            y_{1,R}(t) &= \frac{\sqrt{\delta +\epsilon }}{\sqrt{\delta -\epsilon }} \sinh \left(\frac{T}{2} \sqrt{\delta +\epsilon }\right)\sinh \left(t \sqrt{\delta -\epsilon }\right)+ \cosh \left(\frac{T}{2} \sqrt{\delta +\epsilon }\right)\cosh \left(t \sqrt{\delta -\epsilon }\right) \\
            y_{2,R}(t) &= \frac{1}{\sqrt{\delta -\epsilon }} \cosh \left(\frac{T}{2} \sqrt{\delta +\epsilon }\right)\sinh \left(t \sqrt{\delta -\epsilon }\right)+\frac{1}{\sqrt{\delta +\epsilon }}\sinh \left(\frac{T}{2} \sqrt{\delta +\epsilon }\right)\cosh \left(t \sqrt{\delta -\epsilon }\right) 
        \end{align*}

        We can now compute the Floquet discriminant using 
        
        $$\Gamma = y_1(T) + y_2'(T) = y_{1,R}(T/2) + {y_{2,R}}'(T/2)$$ 
        
        and we find it to be

        \begin{align*}
            \Gamma &= \frac{2 \delta  \sinh \left(\frac{1}{2} T \sqrt{\delta -\epsilon }\right) \sinh \left(\frac{1}{2} T \sqrt{\delta +\epsilon }\right)}{\sqrt{\delta -\epsilon } \sqrt{\delta +\epsilon }}+2 \cosh \left(\frac{1}{2} T \sqrt{\delta -\epsilon }\right) \cosh \left(\frac{1}{2} T \sqrt{\delta +\epsilon }\right)
        \end{align*}

        Lastly, we use $T= 2\pi/\omega$ to write $\Gamma$ in terms of $\omega$, which gives us

        $$\Gamma = \frac{2 \delta  \sinh \left(\frac{\pi  \sqrt{\delta -\epsilon }}{\omega }\right) \sinh \left(\frac{\pi  \sqrt{\delta +\epsilon }}{\omega }\right)}{\sqrt{\delta -\epsilon } \sqrt{\delta +\epsilon }}+2 \cosh \left(\frac{\pi  \sqrt{\delta -\epsilon }}{\omega }\right) \cosh \left(\frac{\pi  \sqrt{\delta +\epsilon }}{\omega }\right)$$

        Hence, given $\delta$, $\epsilon$, and $\omega$ we can determine the stability of the solution by determining whether the Floquet discriminant $\Gamma$ satisfies the condition $$|\Gamma(\delta, \epsilon, \omega)| < 2$$

        \item Evaluate for what values of $\delta$, $\epsilon$, and $\omega$, the pendulum is stabilized. 
        
        \textbf{Solution:} As stated in (a), the stability of the pendulum depends on $|\Gamma(\delta, \epsilon, \omega)|$. There are three primary domains of interest, depending on the relative scale of $\epsilon$ to $\delta$. 

        \begin{enumerate}
            \item $\delta > \epsilon$: In this case all square root terms are real-valued. Hence in the high frequency limit we have 
            
            \begin{align*}
                \lim_{\omega \rightarrow \infty} \Gamma &= \frac{2 \delta  \sinh^2 (0)}{\sqrt{\delta -\epsilon } \sqrt{\delta +\epsilon }}+2 \cosh ^2 (0) =2
            \end{align*}

            So for $\delta > \epsilon$ we have $\Gamma \rightarrow 2$, meaning that the system is on the stability boundary (neutral stability).

            \item $\delta = \epsilon$: In this case the 


        \end{enumerate}

        
    \end{enumerate}
\end{enumerate}

\end{document}