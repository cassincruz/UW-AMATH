\documentclass[10pt]{article}
\usepackage[utf8]{inputenc}
\usepackage{amsmath,amssymb,amsfonts,amsthm}
\usepackage{geometry}
\usepackage{enumitem}
\usepackage{fancyhdr}
\usepackage{lastpage}
\usepackage{graphicx}

% Page layout
\geometry{margin=1in}
\pagestyle{fancy}
\fancyhf{}
\lhead{Lucas Cassin Cruz Burke}
\rhead{AMATH 583}
\rfoot{\thepage}

% Theorem environments
\theoremstyle{definition}
\newtheorem{problem}{Problem}
\theoremstyle{remark}
\newtheorem*{solution}{Solution}
\usepackage[utf8]{inputenc}
\usepackage{geometry}
 \geometry{
 a4paper,
 total={170mm,257mm},
 left=20mm,
 top=20mm,
 }
\usepackage{amsmath}
\usepackage{amsthm}
\usepackage{pdfpages}
\usepackage{enumitem}

\newtheorem{theorem}{Theorem}[section]
\newtheorem{lemma}[theorem]{Lemma}

\usepackage{amssymb}
\usepackage{authblk}
\usepackage{graphicx}
\usepackage{listings}
\usepackage{appendix}
\usepackage[english]{babel}
\usepackage[nottoc]{tocbibind}
\usepackage{fancyhdr}
\usepackage{hyperref}
\hypersetup{
    colorlinks=true,
    linkcolor=blue,
    filecolor=magenta,      
    urlcolor=cyan,
    pdftitle={Overleaf Example},
    pdfpagemode=FullScreen,
    }


%\author{Exam 1}
% \affil{Pacific Northwest National Laboratory, Richland, WA 99352, USA}
% \affil{University of Washington, Seattle, Washington 98195-1560, USA}

%\date{}

\begin{document}
\title{AMATH 483 / 583 - Final}
\author{Due 5pm Friday June 9}
\maketitle
% \tableofcontents

\noindent{\bf Take Home Final (150 points, 20 extra credit points (EC))}
\begin{enumerate}

    % P1
    \item (+20) \textbf{Fourier transforms}. Evaluate the Fourier transform of the following functions by hand. Use the definitions I provided (includes $\frac{1}{\sqrt{2\pi}}$, this is common in physics but also now the default used in WolframAlpha - a powerful math AI tool) as well as the definition for Dirac delta I used if needed.   
    \begin{enumerate}[label=\alph*.]
        \item $f(x) = \frac{1}{\sigma\sqrt{2\pi}} e^{-\frac{1}{2\sigma^2}(x-\mu)^2}$
        \begin{solution}
            \begin{align*}
                \mathcal F\{f(x)\}(k) = F(k) &= \frac{1}{\sqrt{2\pi}} \int_{-\infty}^\infty \frac{1}{\sigma\sqrt{2\pi}} e^{-\frac{1}{2\sigma^2}(x-\mu)^2} e^{i k x} dx \\
                &= \frac{1}{2 \pi \sigma} \int_{-\infty}^\infty \exp\left\{-\frac{1}{2\sigma^2} (x-\mu)^2 + ikx\right\} dx \\
                &= \frac{1}{2\pi \sigma} \exp\left\{ - \frac{\sigma^2k^2}{2} + i k\mu \right\}\int_{-\infty}^\infty \exp\left\{ -\frac{1}{2\sigma^2}(x-\mu-i\sigma^2 k)^2 \right\}dx \\
                &= \frac{1}{2\pi \sigma} \exp\left\{ - \frac{\sigma^2k^2}{2} + i k\mu \right\} \sqrt{2\pi} \sigma \\
                &= \frac{1}{\sqrt{2\pi}}\exp\left\{ - \frac{\sigma^2k^2}{2} + i k\mu \right\}
            \end{align*}
        \end{solution}

        \item $f(t) = sin (\omega_{0} t) \; ,\omega_{0}$ constant
        \begin{solution}
            \begin{align*}
                \mathcal F\{f(t)\}(\omega) = F(\omega) &= \frac{1}{\sqrt{2\pi}} \int_{-\infty}^\infty \sin(\omega_0 t) e^{i \omega t} dt \\
                &= \frac{i}{\sqrt{2\pi}} \int_{-\infty}^\infty \sin(\omega_0 t) \sin(\omega t) dt \\
                &= \frac{i}{\sqrt{2\pi}} \left[ \frac{1}{2} \delta(\omega - \omega_0) - \frac{1}{2} \delta(\omega + \omega_0) \right].
            \end{align*}
        \end{solution}

        \item $f(x) = e^{-a |x|}$ and $a>0$
        \begin{solution}
            \begin{align*}
                \mathcal F\{f(x)\}(k) = F(k) &= \frac{1}{\sqrt{2\pi}} \int_{-\infty}^\infty e^{-a|x|} e^{ikx} dx \\
                &= \frac{1}{\sqrt{2\pi}} \left[ \int_{-\infty}^0 e^{ax + i k x} dx + \int_0^\infty e^{-ax + ikx}dx \right] \\
                &= \frac{1}{\sqrt{2\pi}} \int_0^\infty e^{-ax} \left[ e^{-ikx} + e^{ikx} \right]dx \\
                &= \frac{1}{\sqrt{2\pi}} \left[ \frac{1}{a + ik} + \frac{1}{a - ik}\right] \\&= \sqrt{\frac{2}{\pi}} \frac{a}{a^2 + k^2}
            \end{align*}
        \end{solution}

        \item (distribution) $f(t) = \delta(t)$
        \begin{solution}
            \begin{align*}
                \mathcal F\{f(t)\}(\omega) = F(\omega) &= \frac{1}{\sqrt{2\pi}} \int_{-\infty}^\infty \delta(t) e^{i \omega t} dt \\
                &= \frac{1}{\sqrt{2\pi}}
            \end{align*}
        \end{solution}
    \end{enumerate}

    % P2
    \item (+10) \textbf{Correlation}. By definition, \textit{correlation} is $p\odot q = \frac{1}{\sqrt{2\pi}} \int_{-\infty}^{\infty} p^* (\tau) q(t+\tau) d\tau$, and measures how similar one signal or data function is to another. Let $p(\tau)=\langle p \rangle +\delta_p(\tau)$ and $q(\tau)=\langle q \rangle +\delta_q(\tau)$, where $\langle \rangle$ and $\delta()$ denote the mean values and fluctuation functions (deviations about the mean). Two functions are defined to be \textit{uncorrelated} when $p\odot q = \langle p \rangle \langle q \rangle$. Evaluate $p\odot q$ of the following functions:
    \begin{equation*}
    p(t) = \begin{cases}
    0 & \mbox{ $ t<0 $}\\
    1 & \mbox{ $ 0 < t < 1 $}\\
    0 & \mbox{ $ t > 1 $}
    \end{cases} \quad \; , \;
    q(t) = \begin{cases}
    0 & \mbox{ $ t<0 $}\\
    1-t & \mbox{ $ 0 < t < 1 $}\\
    0 & \mbox{ $ t > 1 $}
    \end{cases}
    \end{equation*}
    \begin{solution}
        For $t >1$ we have $p \odot q = 0$, while for $t \in [0,1]$ we have \begin{align*}
            p \odot q &= \frac{1}{\sqrt{2\pi}} \int_{-\infty}^\infty p^*(\tau)q(t+\tau) d\tau \\
            &= \frac{1}{\sqrt{2\pi}} \int_{0}^{1-t} 1 \cdot (1-t-\tau) d\tau \\
            &= \frac{1}{\sqrt{2\pi}} \left[ \tau - t \tau - \frac{1}{2} \tau^2 \right] \Bigg|_0^{1-t} \\
            &= \frac{1}{\sqrt{2\pi}} \left( 1-t - t(1-t) - \frac{1}{2}(1-t)^2 \right) \\
            &= \frac{1}{2 \sqrt{2 \pi}}
        \end{align*}
    \end{solution}

    % P3
    \item (+5EC) \textbf{Autocorrelation}. Aside, periodic functions exhibit pronounced \textit{autocorrelation}s as shifting such functions by their period puts the function directly on itself. Alternatively, random functions or noise is characterized as being uncorrelated. Evaluate the autocorrelation $p\odot p$ of the following function:
    \[ p(t) = \begin{cases}
    0 & \mbox{ $ t<0 $}\\
    1 & \mbox{ $ 0 < t < 1 $}\\
    0 & \mbox{ $ t > 1 $}
    \end{cases}
    \]

    \begin{solution}
        We have $$p \odot p = \frac{1}{\sqrt{2\pi}} \int_{-\infty}^\infty p^*(\tau) p(t+\tau) d\tau$$

        Plugging in our provided function and looking at where $p(t) \ne 0$, we see that can write this as 

        \begin{align*}
            p \odot p &= \frac{1}{\sqrt{2\pi}} \int_{0}^{1-t} d\tau  \\
            &= \frac{1-t}{\sqrt{2 \pi}}
        \end{align*}
    \end{solution}

    % P4
    \item (+20) \textbf{Fourier transform diffusion equation solve}. Consider the diffusion equation $\frac{\partial T}{\partial t} = \kappa \frac{\partial^2 T}{\partial x^2}$ where $T(x,t)$ describes the temperature profile of a long metal rod. 
    \begin{enumerate}[label=\alph*]
        \item Assume you know $T(x,0)$ and define the Fourier transform of $T(x,t)$ to be $\tau(k,t)$. Transform the original equation and initial conditions into $k$-space. Solve the resulting equation. Inverse transform the result to obtain the solution in terms of the original variables. 
        \begin{solution}
            We begin by transforming our equation into $k$-space. We define $$\tau(k,t) = \frac{1}{\sqrt{2\pi}} \int_{-\infty}^\infty e^{ikx} T(x,t) dx$$

            Then we have $\mathcal F\left\{ \frac{\partial T}{\partial t} \right\} = \frac{\partial \tau}{\partial t}$ and $\mathcal F\left\{ \frac{\partial^2 T}{\partial x^2} \right\} = -k^2 \tau(x,t)$. Hence our transformed equation becomes $$\frac{\partial \tau}{\partial t} = -\kappa k^2 \tau$$

            This is a first order ODE whose solution is given by $$\tau (k,t) = A e^{-\kappa k^2 t}$$

            Let $T_0(x) = T(x,0)$ be our initial condition, and let $\mathcal F\{T_0(x)\} = \hat T_0(k)$ be the Fourier transform of our initial condition. Using this, we can solve for our solution coefficient $A$ and write $$\tau(k,t) = \hat T_0(k) e^{-\kappa k^2 t}$$

            To find our solution in position space in terms of the original variables we take the inverse Fourier transform. Our solution is then given by $$T(x,t) = \mathcal F^{-1} \{ \tau(k,t) \} =  \frac{1}{\sqrt{2\pi}} \int_{-\infty}^\infty \hat T_0(k) e^{-\kappa k^2 t - ikx} dk$$
        \end{solution}

        \item Find the temperature in the rod given initial conditions $\kappa = 10^3 \frac{m^2}{s}$ and 
        \begin{align*}
        T(x,0) = \begin{cases}
        0 & |x| > 1 m \\
        100^{o} C & |x| \le 1 m 
        \end{cases}
        \end{align*}
        \begin{solution}
            We are given $T_0(x) = T(x,0)$ as defined above. We begin by Fourier transforming this initial condition to find $\hat T_0(k)$.
            \begin{align*}
                \hat T_0(k) &= \frac{1}{\sqrt{2\pi}} \int_{-\infty}^\infty T_0(x) e^{ikx} dx \\
                &= \frac{100}{\sqrt{2\pi}} \int_{-1}^1 e^{ikx} dx = \frac{100}{\sqrt{2\pi}} \frac{e^{ikx}}{ik} \Bigg|_{-1}^1 \\&= 100\sqrt{\frac{2}{\pi}} \frac{\sin(k)}{k}
            \end{align*}

            Having found $\hat T_0(k)$, we can plug this in to our inverse Fourier transform expression derived in part (a) to recover our solution $T(x,t)$.

            \begin{align*}
                T(x,t) &= \frac{100}{\pi} \int_{-\infty}^\infty \frac{\sin(k)}{k} e^{-\kappa k^2 t - ikx} dk 
            \end{align*}
        \end{solution}
    \end{enumerate} 

    % P5
    \item (+20) \textbf{Compare OpenBLAS to CUBLAS on HYAK}. Measure and plot the performance of double precision matrix multiply ($\alpha A B + \beta C \rightarrow C$) for square matrices of dimension $n=16$ to $n=8192$, stride $n*=2$ for both the OpenBLAS and CUDA BLAS (CUBLAS) implementations on HYAK. Let each $n$ be measured $ntrial$ times and plot the average performance for each case versus $n$, $ntrial\geq 3$. Submit your performance plot and C++ test code. Your plot will have 'flops' on the y-axis, or some variation of FLOPs such as MFLOPs, and the dimension of the matrices on the x-axis.  
    
    % P6
    \item (+10EC) \textbf{Matrix transpose}. Write C++ functions given the two APIs that compute $A^T$, the matrix transpose of a matrix stored in a single vector container in the column major index scheme. Test the correctness of both functions. Put the functions in file \textbf{transpose.hpp} which I will include in my test code for grading. Submit file \textbf{transpose.hpp}. Note the threaded function will create and join the threads internally. 
    \begin{lstlisting}[language=C++]
    void sequentialTranspose(std::vector<int> &matrix, int rows, int cols);
    void threadedTranspose(std::vector<int> &matrix, int rows, int cols, int nthreads);
    \end{lstlisting}

    % P7
    \item (+20) \textbf{Memory access time}. On a computer of your choice, write C++ functions for the given APIs that perform row and column swap operations in \textit{memory} on a type double matrix stored in column major index order using a single vector container for the data. Test the swap capabilities on randomly selected index pairs using the function I provide here. Put your functions (not 'getRandomIndices') in file \textbf{mem\_swaps.hpp}, no need for header guards -just the code for your functions. Conduct a performance test for square matrix dimensions 16, 32, 64, 128, ... 16384, measuring the time required to conduct row and column swaps separately. Let each operation be measured $ntrial$ times and plot the average time versus matrix dimension, $ntrial\geq 3$. Make a single plot of your \textit{row} and \textit{column} swap timing measurements with time on the y-axis ($log_{10}(time)$) and the problem dimension on the x-axis. Submit file \textbf{mem\_swaps.hpp} and plot. 
    \begin{lstlisting}[language=C++]

    void swapRows(std::vector<double> &matrix, int nRows, int nCols, int i, int j);
    void swapCols(std::vector<double> &matrix, int nRows, int nCols, int i, int j);

    #include <utility>   // For std::pair
    std::pair<int, int> getRandomIndices(int n)
    {
        int i = std::rand() % n;
        int j = std::rand() % (n - 1);
        if (j >= i)
        {
            j++;
        }
        return std::make_pair(i, j);
    }


    // ... from inside main()
    //std::pair<int, int> rowIndices = getRandomIndices(M);
    //int i = rowIndices.first;
    //int j = rowIndices.second;
    //std::pair<int, int> colIndices = getRandomIndices(N);
    // ...

    \end{lstlisting}

    % P8
    \item (+20) \textbf{File access time}. On the same computer as problem 1, write C++ functions for the given APIs that perform row and column swap operations on a type double matrix stored in column major index order in a \textit{FILE}. Use a randomly selected pair of indices to test the swapping capabilities. Put the functions you write in file \textbf{file\_swaps.hpp}. Conduct a performance test for square matrix dimensions 16, 32, 64, 128, ... 16384, measuring the time required to conduct \textit{file-based} row and column swaps separately. Let each operation be measured $ntrial$ times and plot the average time versus matrix dimension, $ntrial\geq 3$. Make a single plot of your \textit{file-based} \textit{row} and \textit{column} swap timing measurements with time on the y-axis ($log_{10}(time)$) and the problem dimension on the x-axis. Submit your header file \textbf{file\_swaps.hpp} and plot. 
    \begin{lstlisting}[language=C++]
    void swapRowsInFile(std::fstream &file, int nRows, int nCols, int i, int j);
    void swapColsInFile(std::fstream &file, int nRows, int nCols, int i, int j);

    // snippet 
    #include <iostream>
    #include <fstream>
    #include <vector>
    #include <utility>
    #include <algorithm>
    #include <cstdlib>
    #include <ctime>
    #include <cstdio>
    #include <chrono>
    #include "file_swaps.hpp"

    int main(int argc, char *argv[])
    {
    // Generate the matrix 
    std::vector<double> matrix(numRows * numCols);
    // init matrix elements in column major order
    // write the matrix to a file
    std::fstream file(filename, std::ios::out | std::ios::binary);
    file.write(reinterpret_cast<char *>(&matrix[0]), numRows * numCols * sizeof(double));
    file.close();
    // Open the file in read-write mode for swapping
    std::fstream fileToSwap(filename, std::ios::in | std::ios::out | std::ios::binary);
    // Get random indices i and j for row swapping
    // Measure the time required for row swapping using file I/O
    auto startTime = std::chrono::high_resolution_clock::now();
    // Swap rows i and j in the file version of the matrix
    swapRowsInFile(fileToSwap, numRows, numCols, i, j);
    auto endTime = std::chrono::high_resolution_clock::now();
    std::chrono::duration<double> duration = endTime - startTime;
    // Close the file after swapping
    fileToSwap.close();
    //...
    // after each problem size delete the test file
    std::remove(filename.c_str());
    // ...
    }
    \end{lstlisting}

    % P9
    \item (+20) \textbf{CPU-GPU data copy speed on HYAK}. Write a C++ code to measure the data copy performance between the host CPU and GPU, and between the GPU and the host CPU. Copy 8 bytes to 256MB increasing in multiples of 2. You will plot the bandwidth for both directions: (bytes per second) on the y-axis, and the buffer size in bytes on the x-axis. Submit your plot and test code.

    % P10
    \item (+20) \textbf{Compare FFTW to CUFFT on HYAK}. Measure and plot the performance of calculating the gradient of a 3D double complex plane wave defined on cubic lattices of dimension $n^3$ from $16^3$ to $n=256^3$, stride $n*=2$ for both the FFTW and CUDA FFT (CUFFT) implementations on HYAK. Let each $n$ be measured $ntrial$ times and plot the average performance for each case versus $n$, $ntrial\geq 3$. Submit your performance plot and C++ test code. Your plot will have 'flops' on the y-axis (or some appropriate unit of FLOPs) and the dimension of the cubic lattices ($n$) on the x-axis. You will need to estimate the operation count of computing the derivative using FFT on a lattice. 

    % P11
    \item (+5EC) \textbf{Root finding}. Write a C++ function that implements a Newton or bisection iteration to estimate the \textit{real} roots to a polynomial equation. Your code should accept the degree of the polynomial, the coefficients, and the domain from the command line (as below). Submit your test code. I will run it against a couple test polynomials of degree 3 and 4 that have simple real roots. 
    \begin{verbatim}
        Enter the degree of the polynomial: 3
        Enter coefficient 3: *
        Enter coefficient 2: *
        Enter coefficient 1: *
        Enter coefficient 0: *
        Enter the start of the domain: 0
        Enter the end of the domain: 10
        Roots found:
        -0.*****
        1.**
        3.*****
    \end{verbatim}
\end{enumerate}


\end{document}